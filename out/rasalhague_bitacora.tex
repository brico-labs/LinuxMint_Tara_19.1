% Options for packages loaded elsewhere
\PassOptionsToPackage{unicode}{hyperref}
\PassOptionsToPackage{hyphens}{url}
\PassOptionsToPackage{dvipsnames,svgnames*,x11names*}{xcolor}
%
\documentclass[
  12pt,
  spanish,
]{article}
\usepackage{lmodern}
\usepackage{amssymb,amsmath}
\usepackage{ifxetex,ifluatex}
\ifnum 0\ifxetex 1\fi\ifluatex 1\fi=0 % if pdftex
  \usepackage[T1]{fontenc}
  \usepackage[utf8]{inputenc}
  \usepackage{textcomp} % provide euro and other symbols
\else % if luatex or xetex
  \usepackage{unicode-math}
  \defaultfontfeatures{Scale=MatchLowercase}
  \defaultfontfeatures[\rmfamily]{Ligatures=TeX,Scale=1}
  \setmainfont[]{Ubuntu}
  \setmonofont[]{Ubuntu Mono}
\fi
% Use upquote if available, for straight quotes in verbatim environments
\IfFileExists{upquote.sty}{\usepackage{upquote}}{}
\IfFileExists{microtype.sty}{% use microtype if available
  \usepackage[]{microtype}
  \UseMicrotypeSet[protrusion]{basicmath} % disable protrusion for tt fonts
}{}
\makeatletter
\@ifundefined{KOMAClassName}{% if non-KOMA class
  \IfFileExists{parskip.sty}{%
    \usepackage{parskip}
  }{% else
    \setlength{\parindent}{0pt}
    \setlength{\parskip}{6pt plus 2pt minus 1pt}}
}{% if KOMA class
  \KOMAoptions{parskip=half}}
\makeatother
\usepackage{xcolor}
\IfFileExists{xurl.sty}{\usepackage{xurl}}{} % add URL line breaks if available
\IfFileExists{bookmark.sty}{\usepackage{bookmark}}{\usepackage{hyperref}}
\hypersetup{
  pdftitle={Bitácora Linux Mint Tara 19.1},
  pdfauthor={Sergio Alvariño salvari@gmail.com},
  colorlinks=true,
  linkcolor=Maroon,
  filecolor=Maroon,
  citecolor=Blue,
  urlcolor=Blue,
  pdfcreator={LaTeX via pandoc}}
\urlstyle{same} % disable monospaced font for URLs
\usepackage[a4paper]{geometry}
\setlength{\emergencystretch}{3em} % prevent overfull lines
\providecommand{\tightlist}{%
  \setlength{\itemsep}{0pt}\setlength{\parskip}{0pt}}
\setcounter{secnumdepth}{5}
\ifxetex
  % Load polyglossia as late as possible: uses bidi with RTL langages (e.g. Hebrew, Arabic)
  \usepackage{polyglossia}
  \setmainlanguage[]{spanish}
\else
  \usepackage[shorthands=off,main=spanish]{babel}
\fi

\title{Bitácora Linux Mint Tara 19.1}
\author{Sergio Alvariño
\href{mailto:salvari@gmail.com}{\nolinkurl{salvari@gmail.com}}}
\date{abril-2019}

\begin{document}
\maketitle
\begin{abstract}
Bitácora de mi portatil

Solo para referencia rápida y personal.
\end{abstract}

{
\hypersetup{linkcolor=}
\setcounter{tocdepth}{3}
\tableofcontents
}
\hypertarget{introducciuxf3n}{%
\section{Introducción}\label{introducciuxf3n}}

Mi portátil es un ordenador Acer 5755G con las siguientes
características:

\begin{itemize}
\item
  Core i5 2430M 2.4GHz
\item
  NVIDIA Geforce GT 540M
\item
  8Gb RAM
\item
  750Gb HD
\end{itemize}

Mi portátil equipa una tarjeta \emph{Nvidia Geforce GT540M} que resulta
pertenecer a una rama muerta en el árbol de desarrollo de Nvidia.

Esta tarjeta provoca todo tipo de problemas de sobrecalientamiento, no
importa que versión de Linux uses.

\hypertarget{programas-buxe1sicos}{%
\section{Programas básicos}\label{programas-buxe1sicos}}

\hypertarget{linux-mint}{%
\subsection{Linux Mint}\label{linux-mint}}

Linux Mint incluye \texttt{sudo} \footnote{ya no incluye gksu pero
  tampoco es imprescindible} y las aplicaciones que uso habitualmente
para gestión de paquetes por defecto (\emph{aptitude} y
\emph{synaptic}).

Tampoco voy a enredar nada con los orígenes del software (de momento)

\hypertarget{firmware}{%
\subsection{Firmware}\label{firmware}}

Instalamos el paquete \texttt{intel-microcode} desde el gestor de
drivers.

Instalamos el driver recomendado de nvidia desde el gestor de drivers
del \emph{Linux Mint}. Ahora mismo es el \emph{nvidia-driver-390}

Configuramos desde el interfaz del driver para activar la tarjeta intel.

Como a pesar de eso seguimos teniendo problemas de calentamiento:

\begin{verbatim}
apt install tlp
tlp start
apt install lm-sensors hddtemp
apt install linux-tools-common linux-tools-generic
cpupower frequency-set -g powersave
apt install cpufrequtils
\end{verbatim}

Referencias:

\begin{itemize}
\tightlist
\item
  \url{https://itsfoss.com/reduce-overheating-laptops-linux/}
\item
  \url{http://www.webupd8.org/2014/04/prevent-your-laptop-from-overheating.html}
\end{itemize}

Después de un reinicio \textbf{frio} \footnote{puede que haya un
  \emph{bug} que hace fallar el sensor de temperatura si el portatil no
  arranca frio} todo parece funcionar de nuevo.

\hypertarget{paruxe1metros-de-disco-duro}{%
\subsection{Parámetros de disco
duro}\label{paruxe1metros-de-disco-duro}}

Tengo un disco duro ssd.

Añadimos el parámetro \texttt{noatime} para las particiones de
\texttt{root} y \texttt{/home}.

\begin{verbatim}
# /etc/fstab: static file system information.
#
# Use 'blkid' to print the universally unique identifier for a
# device; this may be used with UUID= as a more robust way to name devices
# that works even if disks are added and removed. See fstab(5).
#
# <file system> <mount point>   <type>  <options>       <dump>  <pass>
# / was on /dev/sda5 during installation
UUID=d96a5501-75b9-4a25-8ecb-c84cd4a3fff5 /               ext4    noatime,errors=remount-ro 0       1
# /home was on /dev/sda7 during installation
UUID=8fcde9c5-d694-4417-adc0-8dc229299f4c /home           ext4    defaults,noatime        0       2
# /store was on /dev/sdc7 during installation
UUID=0f0892e0-9183-48bd-aab4-9014dc1bd03a /store          ext4    defaults        0       2
# swap was on /dev/sda6 during installation
UUID=ce11ccb0-a67d-4e8b-9456-f49a52974160 none            swap    sw              0       0
# swap was on /dev/sdc5 during installation
UUID=11090d84-ce98-40e2-b7be-dce3f841d7b4 none            swap    sw              0       0
\end{verbatim}

Una vez modificado el \texttt{/etc/fstab} no hace falta arrancar:

\begin{verbatim}
mount -o remount /
mount -o remount /home
mount
\end{verbatim}

En el printado de \texttt{mount} ya veremos si ha cargado el parámetro.

Pasamos el \texttt{fstrim} desde weekly a daily.

Seguimos instrucciones de
\href{https://easylinuxtipsproject.blogspot.com/p/ssd.html}{aquí}.

Más concretamente de
\href{https://easylinuxtipsproject.blogspot.com/p/ssd.html\#ID8.2}{aquí}

y cambiamos el parámetro de \emph{swapiness} a 1.

\hypertarget{fuentes-adicionales}{%
\subsection{Fuentes adicionales}\label{fuentes-adicionales}}

Instalamos algunas fuentes desde los orígenes de software:

\begin{verbatim}
sudo apt install ttf-mscorefonts-installer
sudo apt install fonts-noto
\end{verbatim}

Y la fuente
\href{https://robey.lag.net/2010/06/21/mensch-font.html}{Mensch} la
bajamos directamente al directorio
\texttt{\textasciitilde{}/.local/share/fonts}

\hypertarget{firewall}{%
\subsection{Firewall}\label{firewall}}

\texttt{ufw} y \texttt{gufw} vienen instalados por defecto, pero no
activados.

\begin{verbatim}
aptitude install ufw
ufw default deny
ufw enable
ufw status verbose
aptitude install gufw
\end{verbatim}

\hypertarget{control-de-configuraciones-con-git}{%
\subsection{Control de configuraciones con
git}\label{control-de-configuraciones-con-git}}

\hypertarget{instalaciuxf3n-de-etckeeper}{%
\subsubsection{\texorpdfstring{Instalación de
\texttt{etckeeper}}{Instalación de etckeeper}}\label{instalaciuxf3n-de-etckeeper}}

\begin{verbatim}
sudo su -
git config --global user.email xxxxx@whatever.com
git config --global user.name "Name Surname"
apt install etckeeper
\end{verbatim}

\emph{etckeeper} hara un control automático de tus ficheros de
configuración en \texttt{/etc}

Para echar una mirada a los \emph{commits} creados puedes ejecutar:

\begin{verbatim}
cd /etc
sudo git log
\end{verbatim}

\hypertarget{controlar-dotfiles-con-git}{%
\subsubsection{Controlar dotfiles con
git}\label{controlar-dotfiles-con-git}}

Vamos a crear un repo de git para controlar nuestros ficheros personales
de configuración.

Creamos el repo donde queramos

\begin{verbatim}
mkdir usrcfg
cd usrcfg
git init
git config core.worktree "/home/salvari"
\end{verbatim}

Y ya lo tenemos, un repo que tiene el directorio de trabajo apuntando a
nuestro \emph{\$HOME}.

Podemos añadir los ficheros de configuración que queramos al repo:

\begin{verbatim}
git add .bashrc
git add .zshrc
git commit -m "Add some dotfiles"
\end{verbatim}

Una vez que he añadido los ficheros que quiero tener controlados he
puesto un \texttt{*} en el fichero \texttt{.git/info/exclude} de mi repo
para que ignore todos los ficheros de mi \texttt{\$HOME}.

Cuando instalo algún programa nuevo añado a mano los ficheros de
configuración al repo.

\hypertarget{aplicaciones-variadas}{%
\subsection{Aplicaciones variadas}\label{aplicaciones-variadas}}

\textbf{Nota}: Ya no instalamos \emph{menulibre}, Linux Mint tiene una
utilidad de edición de menús.

\begin{description}
\item[Keepass2]
Para mantener nuestras contraseñas a buen recaudo
\item[Gnucash]
Programa de contabilidad
\item[Deluge]
Programa de descarga de torrents (acuérdate de configurar tus
cortafuegos)
\item[Chromium]
Como Chrome pero libre
\item[rsync, grsync]
Para hacer backups de nuestros ficheros
\item[Descompresores variados]
Para lidiar con los distintos formatos de ficheros comprimidos
\end{description}

\begin{verbatim}
sudo apt install keepass2 gnucash deluge rsync grsync rar unrar \
zip unzip unace bzip2 lzop p7zip p7zip-full p7zip-rar chromium-browser
\end{verbatim}

\hypertarget{programas-de-terminal}{%
\subsection{Programas de terminal}\label{programas-de-terminal}}

Dos imprescindibles:

\begin{verbatim}
sudo apt install guake terminator
\end{verbatim}

\textbf{TODO:} asociar \emph{Guake} a una combinación apropiada de
teclas.

\hypertarget{dropbox}{%
\subsection{Dropbox}\label{dropbox}}

Lo instalamos desde el software manager.

\hypertarget{chrome}{%
\subsection{Chrome}\label{chrome}}

Instalado desde \href{https://www.google.com/chrome/}{la página web de
Chrome}

\hypertarget{varias-aplicaciones-instaladas-de-binarios}{%
\subsection{Varias aplicaciones instaladas de
binarios}\label{varias-aplicaciones-instaladas-de-binarios}}

Lo recomendable en un sistema POSIX es instalar los programas
adicionales en \texttt{/usr/local} o en \texttt{/opt}. Yo soy más
chapuzas y suelo instalar en \texttt{\textasciitilde{}/apt} por que el
portátil es personal e intrasferible. En un ordenador compartido es
mejor usar \texttt{/opt}.

\hypertarget{freeplane}{%
\subsubsection{Freeplane}\label{freeplane}}

Para hacer mapas mentales, presentaciones, resúmenes, apuntes\ldots{} La
versión incluida en LinuxMint está un poco anticuada.

\begin{enumerate}
\def\labelenumi{\arabic{enumi}.}
\tightlist
\item
  descargamos desde
  \href{http://freeplane.sourceforge.net/wiki/index.php/Home}{la web}.
\item
  Descomprimimos en \texttt{\textasciitilde{}/apps/freeplane}
\item
  Creamos enlace simbólico
\item
  Añadimos a los menús
\end{enumerate}

\hypertarget{telegram-desktop}{%
\subsubsection{Telegram Desktop}\label{telegram-desktop}}

Cliente de Telegram, descargado desde la
\href{https://desktop.telegram.org/}{página web}.

\hypertarget{tor-browser}{%
\subsubsection{Tor browser}\label{tor-browser}}

Descargamos desde la \href{https://www.torproject.org/}{página oficial
del proyecto} Descomprimimos en \texttt{\textasciitilde{}/apps/} y
ejecutamos desde terminal:

\begin{verbatim}
cd ~/apps/tor-browser
./start-tor-browser.desktop --register-app
\end{verbatim}

\hypertarget{tiddlydesktop}{%
\subsubsection{TiddlyDesktop}\label{tiddlydesktop}}

Descargamos desde la
\href{https://github.com/Jermolene/TiddlyDesktop}{página web},
descomprimimos y generamos la entrada en el menú.

\hypertarget{terminal-y-shell}{%
\subsection{Terminal y Shell}\label{terminal-y-shell}}

Por defecto tenemos instalado \texttt{bash}.

\hypertarget{bash-git-promt}{%
\subsubsection{bash-git-promt}\label{bash-git-promt}}

Seguimos las instrucciones de
\href{https://github.com/magicmonty/bash-git-prompt}{este github}

\hypertarget{zsh}{%
\subsubsection{zsh}\label{zsh}}

Nos adelantamos a los acontecimientos, pero conviene tener instaladas
las herramientas de entornos virtuales de python antes de instalar
\emph{zsh} con el plugin para \emph{virtualenvwrapper}.

\begin{verbatim}
apt install python-all-dev
apt install python3-all-dev
apt install python-pip python-virtualenv virtualenv python3-pip
\end{verbatim}

\emph{zsh} viene por defecto en mi instalación, en caso contrario:

\begin{verbatim}
apt install zsh
\end{verbatim}

Para \emph{zsh} vamos a usar
\href{https://github.com/zsh-users/antigen}{antigen}, así que nos lo
clonamos en \texttt{\textasciitilde{}/apps/}

\begin{verbatim}
cd ~/apps
git clone https://github.com/zsh-users/antigen
\end{verbatim}

También vamos a usar
\href{https://github.com/olivierverdier/zsh-git-prompt}{zsh-git-prompt},
así que lo clonamos también:

\begin{verbatim}
cd ~/apps
git clone https://github.com/olivierverdier/zsh-git-prompt)
\end{verbatim}

Y editamos el fichero \texttt{\textasciitilde{}/.zshrc} para que
contenga:

\begin{verbatim}
# This line loads .profile, it's experimental
[[ -e ~/.profile ]] && emulate sh -c 'source ~/.profile'

source ~/apps/zsh-git-prompt/zshrc.sh
source ~/apps/antigen/antigen.zsh

# Load the oh-my-zsh's library.
antigen use oh-my-zsh

# Bundles from the default repo (robbyrussell's oh-my-zsh).
antigen bundle git
antigen bundle command-not-found

# must install autojump for this
#antigen bundle autojump

# extracts every kind of compressed file
antigen bundle extract

# jump to dir used frequently
antigen bundle z

#antigen bundle pip

antigen bundle common-aliases

antigen bundle robbyrussell/oh-my-zsh plugins/virtualenvwrapper

antigen bundle zsh-users/zsh-completions

# Syntax highlighting bundle.
antigen bundle zsh-users/zsh-syntax-highlighting
antigen bundle zsh-users/zsh-history-substring-search ./zsh-history-substring-search.zsh

# Arialdo Martini git needs awesome terminal font
#antigen bundle arialdomartini/oh-my-git
#antigen theme arialdomartini/oh-my-git-themes oppa-lana-style

# autosuggestions
antigen bundle tarruda/zsh-autosuggestions

#antigen theme agnoster
antigen theme gnzh

# Tell antigen that you're done.
antigen apply

# Correct rm alias from common-alias bundle
unalias rm
alias rmi='rm -i'
\end{verbatim}

Antigen ya se encarga de descargar todos los plugins que queramos
utilizar en zsh. Todos el software se descarga en
\texttt{\textasciitilde{}/.antigen}

Para configurar el
\href{https://github.com/olivierverdier/zsh-git-prompt}{zsh-git-prompt},
que inspiró el bash-git-prompt, he modificado el fichero
\texttt{\textasciitilde{}/.zshrc} y el fichero del tema en
\texttt{\textasciitilde{}/.antigen/bundles/robbyrussell/oh-my-zsh/themes/gnzh.zsh-theme}

\hypertarget{fish}{%
\subsubsection{fish}\label{fish}}

\textbf{Nota}: No he instalado \emph{fish} dejo por aquí las notas del
antiguo linux mint por si le interesa a alguien.

Instalamos \emph{fish}:

\begin{verbatim}
sudo aptitude install fish
\end{verbatim}

Instalamos oh-my-fish

\begin{verbatim}
curl -L https://github.com/oh-my-fish/oh-my-fish/raw/master/bin/install > install
fish install
rm install
\end{verbatim}

Si queremos que fish sea nuestro nuevo shell:

\begin{verbatim}
chsh -s `which fish`
\end{verbatim}

Los ficheros de configuración de \emph{fish} se encuentran en
\texttt{\textasciitilde{}/config/fish}.

Los ficheros de \emph{Oh-my-fish} en mi portátil quedan en
\texttt{\textasciitilde{}/.local/share/omf}

Para tener la info de git en el prompt de fish al estilo de
\href{https://github.com/magicmonty/bash-git-prompt}{bash-git-prompt},
copiamos:

\begin{verbatim}
cp ~/.bash-git-prompt/gitprompt.fish ~/.config/fish/functions/fish_prompt.fish
\end{verbatim}

\textbf{NOTA}: \emph{fish} es un shell estupendo supercómodo con un
montón de funcionalidades. Pero no es POSIX. Mucho ojo con esto, usa
\emph{fish} pero aségurate de saber a que renuncias, o las
complicaciones a las que vas a enfrentarte.

\hypertarget{tmux}{%
\subsubsection{tmux}\label{tmux}}

Esto no tiene mucho que ver con los shell, lo he instalado para aprender
a usarlo.

\begin{verbatim}
sudo apt install tmux
\end{verbatim}

\hypertarget{utilidades}{%
\subsection{Utilidades}\label{utilidades}}

\emph{Agave} y \emph{pdftk} ya no existen, nos pasamos a \emph{gpick} y
\emph{poppler-utils}:

Instalamos \emph{gpick} con \texttt{sudo\ apt\ install\ gpick}

\hypertarget{codecs}{%
\subsection{Codecs}\label{codecs}}

\begin{verbatim}
sudo apt-get install mint-meta-codecs
\end{verbatim}

\end{document}

% Options for packages loaded elsewhere
\PassOptionsToPackage{unicode}{hyperref}
\PassOptionsToPackage{hyphens}{url}
\PassOptionsToPackage{dvipsnames,svgnames*,x11names*}{xcolor}
%
\documentclass[
  12pt,
  spanish,
]{article}
\usepackage{lmodern}
\usepackage{amssymb,amsmath}
\usepackage{ifxetex,ifluatex}
\ifnum 0\ifxetex 1\fi\ifluatex 1\fi=0 % if pdftex
  \usepackage[T1]{fontenc}
  \usepackage[utf8]{inputenc}
  \usepackage{textcomp} % provide euro and other symbols
\else % if luatex or xetex
  \usepackage{unicode-math}
  \defaultfontfeatures{Scale=MatchLowercase}
  \defaultfontfeatures[\rmfamily]{Ligatures=TeX,Scale=1}
  \setmainfont[]{Ubuntu}
  \setmonofont[]{Ubuntu Mono}
\fi
% Use upquote if available, for straight quotes in verbatim environments
\IfFileExists{upquote.sty}{\usepackage{upquote}}{}
\IfFileExists{microtype.sty}{% use microtype if available
  \usepackage[]{microtype}
  \UseMicrotypeSet[protrusion]{basicmath} % disable protrusion for tt fonts
}{}
\makeatletter
\@ifundefined{KOMAClassName}{% if non-KOMA class
  \IfFileExists{parskip.sty}{%
    \usepackage{parskip}
  }{% else
    \setlength{\parindent}{0pt}
    \setlength{\parskip}{6pt plus 2pt minus 1pt}}
}{% if KOMA class
  \KOMAoptions{parskip=half}}
\makeatother
\usepackage{xcolor}
\IfFileExists{xurl.sty}{\usepackage{xurl}}{} % add URL line breaks if available
\IfFileExists{bookmark.sty}{\usepackage{bookmark}}{\usepackage{hyperref}}
\hypersetup{
  pdftitle={Bitácora Linux Mint Tara 19.1},
  pdfauthor={Sergio Alvariño salvari@gmail.com},
  colorlinks=true,
  linkcolor=Maroon,
  filecolor=Maroon,
  citecolor=Blue,
  urlcolor=Blue,
  pdfcreator={LaTeX via pandoc}}
\urlstyle{same} % disable monospaced font for URLs
\usepackage[a4paper]{geometry}
\usepackage{longtable,booktabs}
% Correct order of tables after \paragraph or \subparagraph
\usepackage{etoolbox}
\makeatletter
\patchcmd\longtable{\par}{\if@noskipsec\mbox{}\fi\par}{}{}
\makeatother
% Allow footnotes in longtable head/foot
\IfFileExists{footnotehyper.sty}{\usepackage{footnotehyper}}{\usepackage{footnote}}
\makesavenoteenv{longtable}
\setlength{\emergencystretch}{3em} % prevent overfull lines
\providecommand{\tightlist}{%
  \setlength{\itemsep}{0pt}\setlength{\parskip}{0pt}}
\setcounter{secnumdepth}{5}
\ifxetex
  % Load polyglossia as late as possible: uses bidi with RTL langages (e.g. Hebrew, Arabic)
  \usepackage{polyglossia}
  \setmainlanguage[]{spanish}
\else
  \usepackage[shorthands=off,main=spanish]{babel}
\fi

\title{Bitácora Linux Mint Tara 19.1}
\author{Sergio Alvariño
\href{mailto:salvari@gmail.com}{\nolinkurl{salvari@gmail.com}}}
\date{abril-2019}

\begin{document}
\maketitle
\begin{abstract}
Bitácora de mi portatil

Solo para referencia rápida y personal.
\end{abstract}

{
\hypersetup{linkcolor=}
\setcounter{tocdepth}{3}
\tableofcontents
}
\hypertarget{introducciuxf3n}{%
\section{Introducción}\label{introducciuxf3n}}

Mi portátil es un ordenador Acer 5755G con las siguientes
características:

\begin{itemize}
\item
  Core i5 2430M 2.4GHz
\item
  NVIDIA Geforce GT 540M
\item
  8Gb RAM
\item
  750Gb HD
\end{itemize}

Mi portátil equipa una tarjeta \emph{Nvidia Geforce GT540M} que resulta
pertenecer a una rama muerta en el árbol de desarrollo de Nvidia.

Esta tarjeta provoca todo tipo de problemas de sobrecalientamiento, no
importa que versión de Linux uses.

\hypertarget{programas-buxe1sicos}{%
\section{Programas básicos}\label{programas-buxe1sicos}}

\hypertarget{linux-mint}{%
\subsection{Linux Mint}\label{linux-mint}}

Linux Mint incluye \texttt{sudo} \footnote{ya no incluye gksu pero
  tampoco es imprescindible} y las aplicaciones que uso habitualmente
para gestión de paquetes por defecto (\emph{aptitude} y
\emph{synaptic}).

Tampoco voy a enredar nada con los orígenes del software (de momento)

\hypertarget{firmware}{%
\subsection{Firmware}\label{firmware}}

Instalamos el paquete \texttt{intel-microcode} desde el gestor de
drivers.

Instalamos el driver recomendado de nvidia desde el gestor de drivers
del \emph{Linux Mint}. Ahora mismo es el \emph{nvidia-driver-390}

Configuramos desde el interfaz del driver para activar la tarjeta intel.

Como a pesar de eso seguimos teniendo problemas de calentamiento:

\begin{verbatim}
apt install tlp
tlp start
apt install lm-sensors hddtemp
apt install linux-tools-common linux-tools-generic
cpupower frequency-set -g powersave
apt install cpufrequtils
\end{verbatim}

Referencias:

\begin{itemize}
\tightlist
\item
  \url{https://itsfoss.com/reduce-overheating-laptops-linux/}
\item
  \url{http://www.webupd8.org/2014/04/prevent-your-laptop-from-overheating.html}
\end{itemize}

Después de un reinicio \textbf{frio} \footnote{puede que haya un
  \emph{bug} que hace fallar el sensor de temperatura si el portatil no
  arranca frio} todo parece funcionar de nuevo.

\hypertarget{paruxe1metros-de-disco-duro}{%
\subsection{Parámetros de disco
duro}\label{paruxe1metros-de-disco-duro}}

Tengo un disco duro ssd.

Añadimos el parámetro \texttt{noatime} para las particiones de
\texttt{root} y \texttt{/home}.

\begin{verbatim}
# /etc/fstab: static file system information.
#
# Use 'blkid' to print the universally unique identifier for a
# device; this may be used with UUID= as a more robust way to name devices
# that works even if disks are added and removed. See fstab(5).
#
# <file system> <mount point>   <type>  <options>       <dump>  <pass>
# / was on /dev/sda5 during installation
UUID=d96a5501-75b9-4a25-8ecb-c84cd4a3fff5 /               ext4    noatime,errors=remount-ro 0       1
# /home was on /dev/sda7 during installation
UUID=8fcde9c5-d694-4417-adc0-8dc229299f4c /home           ext4    defaults,noatime        0       2
# /store was on /dev/sdc7 during installation
UUID=0f0892e0-9183-48bd-aab4-9014dc1bd03a /store          ext4    defaults        0       2
# swap was on /dev/sda6 during installation
UUID=ce11ccb0-a67d-4e8b-9456-f49a52974160 none            swap    sw              0       0
# swap was on /dev/sdc5 during installation
UUID=11090d84-ce98-40e2-b7be-dce3f841d7b4 none            swap    sw              0       0
\end{verbatim}

Una vez modificado el \texttt{/etc/fstab} no hace falta arrancar:

\begin{verbatim}
mount -o remount /
mount -o remount /home
mount
\end{verbatim}

En el printado de \texttt{mount} ya veremos si ha cargado el parámetro.

Pasamos el \texttt{fstrim} desde weekly a daily.

Seguimos instrucciones de
\href{https://easylinuxtipsproject.blogspot.com/p/ssd.html}{aquí}.

Más concretamente de
\href{https://easylinuxtipsproject.blogspot.com/p/ssd.html\#ID8.2}{aquí}

y cambiamos el parámetro de \emph{swapiness} a 1.

\hypertarget{fuentes-adicionales}{%
\subsection{Fuentes adicionales}\label{fuentes-adicionales}}

Instalamos algunas fuentes desde los orígenes de software:

\begin{verbatim}
sudo apt install ttf-mscorefonts-installer
sudo apt install fonts-noto
\end{verbatim}

Y la fuente
\href{https://robey.lag.net/2010/06/21/mensch-font.html}{Mensch} la
bajamos directamente al directorio
\texttt{\textasciitilde{}/.local/share/fonts}

\hypertarget{firewall}{%
\subsection{Firewall}\label{firewall}}

\texttt{ufw} y \texttt{gufw} vienen instalados por defecto, pero no
activados.

\begin{verbatim}
aptitude install ufw
ufw default deny
ufw enable
ufw status verbose
aptitude install gufw
\end{verbatim}

\hypertarget{control-de-configuraciones-con-git}{%
\subsection{Control de configuraciones con
git}\label{control-de-configuraciones-con-git}}

\hypertarget{instalaciuxf3n-de-etckeeper}{%
\subsubsection{\texorpdfstring{Instalación de
\texttt{etckeeper}}{Instalación de etckeeper}}\label{instalaciuxf3n-de-etckeeper}}

\begin{verbatim}
sudo su -
git config --global user.email xxxxx@whatever.com
git config --global user.name "Name Surname"
apt install etckeeper
\end{verbatim}

\emph{etckeeper} hara un control automático de tus ficheros de
configuración en \texttt{/etc}

Para echar una mirada a los \emph{commits} creados puedes ejecutar:

\begin{verbatim}
cd /etc
sudo git log
\end{verbatim}

\hypertarget{controlar-dotfiles-con-git}{%
\subsubsection{Controlar dotfiles con
git}\label{controlar-dotfiles-con-git}}

Vamos a crear un repo de git para controlar nuestros ficheros personales
de configuración.

Creamos el repo donde queramos

\begin{verbatim}
mkdir usrcfg
cd usrcfg
git init
git config core.worktree "/home/salvari"
\end{verbatim}

Y ya lo tenemos, un repo que tiene el directorio de trabajo apuntando a
nuestro \emph{\$HOME}.

Podemos añadir los ficheros de configuración que queramos al repo:

\begin{verbatim}
git add .bashrc
git add .zshrc
git commit -m "Add some dotfiles"
\end{verbatim}

Una vez que he añadido los ficheros que quiero tener controlados he
puesto un \texttt{*} en el fichero \texttt{.git/info/exclude} de mi repo
para que ignore todos los ficheros de mi \texttt{\$HOME}.

Cuando instalo algún programa nuevo añado a mano los ficheros de
configuración al repo.

\hypertarget{aplicaciones-variadas}{%
\subsection{Aplicaciones variadas}\label{aplicaciones-variadas}}

\textbf{Nota}: Ya no instalamos \emph{menulibre}, Linux Mint tiene una
utilidad de edición de menús.

\begin{description}
\item[Keepass2]
Para mantener nuestras contraseñas a buen recaudo
\item[Gnucash]
Programa de contabilidad
\item[Deluge]
Programa de descarga de torrents (acuérdate de configurar tus
cortafuegos)
\item[Chromium]
Como Chrome pero libre
\item[rsync, grsync]
Para hacer backups de nuestros ficheros
\item[Descompresores variados]
Para lidiar con los distintos formatos de ficheros comprimidos
\end{description}

\begin{verbatim}
sudo apt install keepass2 gnucash deluge rsync grsync rar unrar \
zip unzip unace bzip2 lzop p7zip p7zip-full p7zip-rar chromium-browser
\end{verbatim}

\hypertarget{programas-de-terminal}{%
\subsection{Programas de terminal}\label{programas-de-terminal}}

Dos imprescindibles:

\begin{verbatim}
sudo apt install guake terminator
\end{verbatim}

\textbf{TODO:} asociar \emph{Guake} a una combinación apropiada de
teclas.

\hypertarget{dropbox}{%
\subsection{Dropbox}\label{dropbox}}

Lo instalamos desde el software manager.

\hypertarget{chrome}{%
\subsection{Chrome}\label{chrome}}

Instalado desde \href{https://www.google.com/chrome/}{la página web de
Chrome}

\hypertarget{varias-aplicaciones-instaladas-de-binarios}{%
\subsection{Varias aplicaciones instaladas de
binarios}\label{varias-aplicaciones-instaladas-de-binarios}}

Lo recomendable en un sistema POSIX es instalar los programas
adicionales en \texttt{/usr/local} o en \texttt{/opt}. Yo soy más
chapuzas y suelo instalar en \texttt{\textasciitilde{}/apt} por que el
portátil es personal e intrasferible. En un ordenador compartido es
mejor usar \texttt{/opt}.

\hypertarget{freeplane}{%
\subsubsection{Freeplane}\label{freeplane}}

Para hacer mapas mentales, presentaciones, resúmenes, apuntes\ldots{} La
versión incluida en LinuxMint está un poco anticuada.

\begin{enumerate}
\def\labelenumi{\arabic{enumi}.}
\tightlist
\item
  descargamos desde
  \href{http://freeplane.sourceforge.net/wiki/index.php/Home}{la web}.
\item
  Descomprimimos en \texttt{\textasciitilde{}/apps/freeplane}
\item
  Creamos enlace simbólico
\item
  Añadimos a los menús
\end{enumerate}

\hypertarget{telegram-desktop}{%
\subsubsection{Telegram Desktop}\label{telegram-desktop}}

Cliente de Telegram, descargado desde la
\href{https://desktop.telegram.org/}{página web}.

\hypertarget{tor-browser}{%
\subsubsection{Tor browser}\label{tor-browser}}

Descargamos desde la \href{https://www.torproject.org/}{página oficial
del proyecto} Descomprimimos en \texttt{\textasciitilde{}/apps/} y
ejecutamos desde terminal:

\begin{verbatim}
cd ~/apps/tor-browser
./start-tor-browser.desktop --register-app
\end{verbatim}

\hypertarget{tiddlydesktop}{%
\subsubsection{TiddlyDesktop}\label{tiddlydesktop}}

Descargamos desde la
\href{https://github.com/Jermolene/TiddlyDesktop}{página web},
descomprimimos y generamos la entrada en el menú.

\hypertarget{terminal-y-shell}{%
\subsection{Terminal y Shell}\label{terminal-y-shell}}

Por defecto tenemos instalado \texttt{bash}.

\hypertarget{bash-git-promt}{%
\subsubsection{bash-git-promt}\label{bash-git-promt}}

Seguimos las instrucciones de
\href{https://github.com/magicmonty/bash-git-prompt}{este github}

\hypertarget{zsh}{%
\subsubsection{zsh}\label{zsh}}

Nos adelantamos a los acontecimientos, pero conviene tener instaladas
las herramientas de entornos virtuales de python antes de instalar
\emph{zsh} con el plugin para \emph{virtualenvwrapper}.

\begin{verbatim}
apt install python-all-dev
apt install python3-all-dev
apt install python-pip python-virtualenv virtualenv python3-pip
\end{verbatim}

\emph{zsh} viene por defecto en mi instalación, en caso contrario:

\begin{verbatim}
apt install zsh
\end{verbatim}

Para \emph{zsh} vamos a usar
\href{https://github.com/zsh-users/antigen}{antigen}, así que nos lo
clonamos en \texttt{\textasciitilde{}/apps/}

\begin{verbatim}
cd ~/apps
git clone https://github.com/zsh-users/antigen
\end{verbatim}

También vamos a usar
\href{https://github.com/olivierverdier/zsh-git-prompt}{zsh-git-prompt},
así que lo clonamos también:

\begin{verbatim}
cd ~/apps
git clone https://github.com/olivierverdier/zsh-git-prompt)
\end{verbatim}

Y editamos el fichero \texttt{\textasciitilde{}/.zshrc} para que
contenga:

\begin{verbatim}
# This line loads .profile, it's experimental
[[ -e ~/.profile ]] && emulate sh -c 'source ~/.profile'

source ~/apps/zsh-git-prompt/zshrc.sh
source ~/apps/antigen/antigen.zsh

# Load the oh-my-zsh's library.
antigen use oh-my-zsh

# Bundles from the default repo (robbyrussell's oh-my-zsh).
antigen bundle git
antigen bundle command-not-found

# must install autojump for this
#antigen bundle autojump

# extracts every kind of compressed file
antigen bundle extract

# jump to dir used frequently
antigen bundle z

#antigen bundle pip

antigen bundle common-aliases

antigen bundle robbyrussell/oh-my-zsh plugins/virtualenvwrapper

antigen bundle zsh-users/zsh-completions

# Syntax highlighting bundle.
antigen bundle zsh-users/zsh-syntax-highlighting
antigen bundle zsh-users/zsh-history-substring-search ./zsh-history-substring-search.zsh

# Arialdo Martini git needs awesome terminal font
#antigen bundle arialdomartini/oh-my-git
#antigen theme arialdomartini/oh-my-git-themes oppa-lana-style

# autosuggestions
antigen bundle tarruda/zsh-autosuggestions

#antigen theme agnoster
antigen theme gnzh

# Tell antigen that you're done.
antigen apply

# Correct rm alias from common-alias bundle
unalias rm
alias rmi='rm -i'
\end{verbatim}

Antigen ya se encarga de descargar todos los plugins que queramos
utilizar en zsh. Todos el software se descarga en
\texttt{\textasciitilde{}/.antigen}

Para configurar el
\href{https://github.com/olivierverdier/zsh-git-prompt}{zsh-git-prompt},
que inspiró el bash-git-prompt, he modificado el fichero
\texttt{\textasciitilde{}/.zshrc} y el fichero del tema en
\texttt{\textasciitilde{}/.antigen/bundles/robbyrussell/oh-my-zsh/themes/gnzh.zsh-theme}

\hypertarget{fish}{%
\subsubsection{fish}\label{fish}}

\textbf{Nota}: No he instalado \emph{fish} dejo por aquí las notas del
antiguo linux mint por si le interesa a alguien.

Instalamos \emph{fish}:

\begin{verbatim}
sudo aptitude install fish
\end{verbatim}

Instalamos oh-my-fish

\begin{verbatim}
curl -L https://github.com/oh-my-fish/oh-my-fish/raw/master/bin/install > install
fish install
rm install
\end{verbatim}

Si queremos que fish sea nuestro nuevo shell:

\begin{verbatim}
chsh -s `which fish`
\end{verbatim}

Los ficheros de configuración de \emph{fish} se encuentran en
\texttt{\textasciitilde{}/config/fish}.

Los ficheros de \emph{Oh-my-fish} en mi portátil quedan en
\texttt{\textasciitilde{}/.local/share/omf}

Para tener la info de git en el prompt de fish al estilo de
\href{https://github.com/magicmonty/bash-git-prompt}{bash-git-prompt},
copiamos:

\begin{verbatim}
cp ~/.bash-git-prompt/gitprompt.fish ~/.config/fish/functions/fish_prompt.fish
\end{verbatim}

\textbf{NOTA}: \emph{fish} es un shell estupendo supercómodo con un
montón de funcionalidades. Pero no es POSIX. Mucho ojo con esto, usa
\emph{fish} pero aségurate de saber a que renuncias, o las
complicaciones a las que vas a enfrentarte.

\hypertarget{tmux}{%
\subsubsection{tmux}\label{tmux}}

Esto no tiene mucho que ver con los shell, lo he instalado para aprender
a usarlo.

\begin{verbatim}
sudo apt install tmux
\end{verbatim}

\hypertarget{utilidades}{%
\subsection{Utilidades}\label{utilidades}}

\emph{Agave} y \emph{pdftk} ya no existen, nos pasamos a \emph{gpick} y
\emph{poppler-utils}:

Instalamos \emph{gpick} con \texttt{sudo\ apt\ install\ gpick}

\hypertarget{codecs}{%
\subsection{Codecs}\label{codecs}}

\begin{verbatim}
sudo apt-get install mint-meta-codecs
\end{verbatim}

\hypertarget{utilidades-1}{%
\section{Utilidades}\label{utilidades-1}}

\hypertarget{htop}{%
\subsection{htop}\label{htop}}

\begin{verbatim}
sudo apt install htop
\end{verbatim}

\hypertarget{gparted}{%
\subsection{gparted}\label{gparted}}

Instalamos \emph{gparted} para poder formatear memorias usb

\texttt{sudo\ apt\ install\ gparted}

\hypertarget{wkhtmltopdf}{%
\subsection{wkhtmltopdf}\label{wkhtmltopdf}}

\begin{verbatim}
sudo apt install wkhtmltopdf
\end{verbatim}

\hypertarget{internet}{%
\section{Internet}\label{internet}}

\hypertarget{rclone}{%
\section{Rclone}\label{rclone}}

Instalamos desde la página web, siempre que te fies obviamente.

\begin{verbatim}
curl https://rclone.org/install.sh | sudo bash
\end{verbatim}

\hypertarget{recetas-rclone}{%
\subsection{Recetas rclone}\label{recetas-rclone}}

Copiar directorio local en la nube:

\begin{verbatim}
rclone copy /localdir hubic:backup -vv
\end{verbatim}

Si queremos ver el directorio en la web de Hubic tenemos que copiarlo en
\emph{default}:

\begin{verbatim}
rclone copy /localdir hubic:default/backup -vv
\end{verbatim}

Sincronizar una carpeta remota en local:

\begin{verbatim}
rclone sync hubic:directorio_remoto /home/salvari/directorio_local -vv
\end{verbatim}

\hypertarget{referencias}{%
\subsection{Referencias}\label{referencias}}

\begin{itemize}
\tightlist
\item
  \href{https://elblogdelazaro.gitlab.io//articles/rclone-sincroniza-ficheros-en-la-nube/}{Como
  usar rclone (blogdelazaro)}
\item
  \href{https://elblogdelazaro.gitlab.io//articles/rclone-cifrado-de-ficheros-en-la-nube/}{y
  con cifrado (blogdelazaro)}
\item
  \href{https://rclone.org/docs/}{Documentación}
\end{itemize}

\hypertarget{tareas}{%
\section{Tareas}\label{tareas}}

\hypertarget{hamster-indicator}{%
\subsection{hamster-indicator}\label{hamster-indicator}}

Tan fácil como:

\begin{verbatim}
sudo apt install hamster-indicator
\end{verbatim}

\hypertarget{documentaciuxf3n}{%
\section{Documentación}\label{documentaciuxf3n}}

\hypertarget{vanilla-latex}{%
\subsection{Vanilla LaTeX}\label{vanilla-latex}}

El LaTeX de Debian está un poquillo anticuado, si se quiere usar una
versión reciente hay que aplicar este truco.

\begin{verbatim}
cd ~
mkdir tmp
cd tmp
wget http://mirror.ctan.org/systems/texlive/tlnet/install-tl-unx.tar.gz
tar xzf install-tl-unx.tar.gz
cd install-tl-xxxxxx
\end{verbatim}

La parte xxxxxx varía en función del estado de la última versión de
LaTeX disponible.

\begin{verbatim}
sudo ./install-tl
\end{verbatim}

Una vez lanzada la instalación podemos desmarcar las opciones que
instalan la documentación y las fuentes. Eso nos obligará a consultar la
documentación on line pero ahorrará practicamente el 50\% del espacio
necesario. En mi caso sin doc ni src ocupa 2,3Gb

\begin{verbatim}
mkdir -p /opt/texbin
sudo ln -s /usr/local/texlive/2018/bin/x86_64-linux/* /opt/texbin
\end{verbatim}

Por último para acabar la instalación añadimos \texttt{/opt/texbin} al
\emph{PATH}. Para \emph{bash} y \emph{zsh} basta con añadir al fichero
\texttt{\textasciitilde{}/.profile} las siguientes lineas:

\begin{verbatim}
# adds texlive to my PATH
if [ -d "/opt/texbin" ] ; then
    PATH="$PATH:/opt/texbin"
fi
\end{verbatim}

En cuanto a \emph{fish} (si es que lo usas, claro) tendremos que
modificar (o crear) el fichero
\texttt{\textasciitilde{}/.config/fish/config.fish} y añadir la
siguiente linea:

\begin{verbatim}
set PATH $PATH /opt/texbin
\end{verbatim}

\hypertarget{falsificando-paquetes}{%
\subsubsection{Falsificando paquetes}\label{falsificando-paquetes}}

Ya tenemos el \emph{texlive} instalado, ahora necesitamos que el gestor
de paquetes sepa que ya lo tenemos instalado.

\begin{verbatim}
sudo apt install equivs --no-install-recommends
mkdir -p /tmp/tl-equivs && cd /tmp/tl-equivs
equivs-control texlive-local
\end{verbatim}

Alternativamente para hacerlo más fácil podemos descargarnos un fichero
\texttt{texlive-local}ya preparado, ejecutando:

\begin{verbatim}
wget http://www.tug.org/texlive/files/debian-equivs-2018-ex.txt
/bin/cp -f debian-equivs-2018-ex.txt texlive-local
\end{verbatim}

Editamos la versión (si queremos) y procedemos a generar el paquete
\emph{deb}.

\begin{verbatim}
equivs-build texlive-local
\end{verbatim}

El paquete que hemos generado tiene una dependencia: \emph{freeglut3},
hay que instalarla previamente.

\begin{verbatim}
sudo apt install freeglut3
sudo dpkg -i texlive-local_2018-1_all.deb
\end{verbatim}

Todo listo, ahora podemos instalar cualquier paquete debian que dependa
de \emph{texlive} sin problemas de dependencias, aunque no hayamos
instalado el \emph{texlive} de Debian.

\hypertarget{fuentes}{%
\subsubsection{Fuentes}\label{fuentes}}

Para dejar disponibles las fuentes opentype y truetype que vienen con
texlive para el resto de aplicaciones:

\begin{verbatim}
sudo cp $(kpsewhich -var-value TEXMFSYSVAR)/fonts/conf/texlive-fontconfig.conf /etc/fonts/conf.d/09-texlive.conf
gksudo gedit /etc/fonts/conf.d/09-texlive.conf
\end{verbatim}

Borramos la linea:

\begin{verbatim}
<dir>/usr/local/texlive/2018/texmf-dist/fonts/type1</dir>
\end{verbatim}

Y ejecutamos:

\begin{verbatim}
sudo fc-cache -fsv
\end{verbatim}

Actualizaciones Para actualizar nuestro \emph{latex} a la última versión
de todos los paquetes:

\begin{verbatim}
sudo /opt/texbin/tlmgr update --self
sudo /opt/texbin/tlmgr update --all
\end{verbatim}

También podemos lanzar el instalador gráfico con:

\begin{verbatim}
sudo /opt/texbin/tlmgr --gui
\end{verbatim}

Para usar el instalador gráfico hay que instalar previamente:

\begin{verbatim}
sudo apt-get install perl-tk --no-install-recommends
\end{verbatim}

Lanzador para el actualizador de \emph{texlive}:

\begin{verbatim}
mkdir -p ~/.local/share/applications
/bin/rm ~/.local/share/applications/tlmgr.desktop
cat > ~/.local/share/applications/tlmgr.desktop << EOF
[Desktop Entry]
Version=1.0
Name=TeX Live Manager
Comment=Manage TeX Live packages
GenericName=Package Manager
Exec=gksu -d -S -D "TeX Live Manager" '/opt/texbin/tlmgr -gui'
Terminal=false
Type=Application
Icon=system-software-update
EOF
\end{verbatim}

\hypertarget{tipos-de-letra}{%
\subsection{Tipos de letra}\label{tipos-de-letra}}

Creamos el directorio de usuario para tipos de letra:

\begin{verbatim}
mkdir ~/.local/share/fonts
\end{verbatim}

\hypertarget{fuentes-adicionales-1}{%
\subsection{Fuentes Adicionales}\label{fuentes-adicionales-1}}

Un par de fuentes las he descargado de internet y las he almacenado en
el directorio de usuario para los tipos de letra:
\texttt{\textasciitilde{}/.local/share/fonts}

\begin{itemize}
\tightlist
\item
  \href{https://design.ubuntu.com/font/}{Ubuntu} (La uso en
  documentación)
\item
  \href{https://robey.lag.net/downloads/mensch.ttf}{Mensch} (Esta es la
  que yo uso para programar.)
\end{itemize}

Además he clonado el repo
\href{https://github.com/ProgrammingFonts/ProgrammingFonts}{\emph{Programming
Fonts}} y enlazado algunas fuentes (Hack y Menlo)

\begin{verbatim}
cd ~/wherever
git clone https://github.com/ProgrammingFonts/ProgrammingFonts
cd ~/.local/share/fonts
ln -s ~/wherever/ProgrammingFonts/Hack .
ln -s ~/wherever/ProgrammingFonts/Menlo .
\end{verbatim}

\hypertarget{pandoc}{%
\subsection{Pandoc}\label{pandoc}}

\emph{Pandoc} es un traductor entre formatos de documento. Está escrito
en Python y es increiblemente útil. De hecho este documento está escrito
con \emph{Pandoc}.

Instalado el \emph{Pandoc} descargando paquete deb desde
\href{http://pandoc.org/installing.html}{la página web del proyecto}.

Además descargamos plantillas adicionales desde
\href{https://github.com/jgm/pandoc-templates}{este repo} ejecutando los
siguientes comandos:

\begin{verbatim}
mkdir ~/.pandoc
cd ~/.pandoc
git clone https://github.com/jgm/pandoc-templates templates
\end{verbatim}

\hypertarget{calibre}{%
\subsection{Calibre}\label{calibre}}

La mejor utilidad para gestionar tu colección de libros electrónicos.

Ejecutamos lo que manda la página web:

\begin{verbatim}
sudo -v && wget -nv -O- https://raw.githubusercontent.com/kovidgoyal/calibre/master/setup/linux-installer.py \
| sudo python -c "import sys; main=lambda:sys.stderr.write('Download failed\n'); exec(sys.stdin.read()); main()"
\end{verbatim}

Para usar el calibre con el Kobo Glo:

\begin{itemize}
\tightlist
\item
  Desactivamos todos los plugin de Kobo menos el Kobo Touch Extended
\item
  Creamos una columna MyShelves con identificativo \#myshelves
\item
  En las opciones del plugin:

  \begin{itemize}
  \tightlist
  \item
    En la opción Collection columns añadimos las columnas
    series,\#myshelves
  \item
    Marcamos las opciones Create collections y Delete empy collections
  \item
    Marcamos \emph{Modify CSS}
  \item
    Update metadata on device y Set series information
  \end{itemize}
\end{itemize}

Algunos enlaces útiles:

\begin{itemize}
\tightlist
\item
  (https://github.com/jgoguen/calibre-kobo-driver)
\item
  (http://www.lectoreselectronicos.com/foro/showthread.php?15116-Manual-de-instalaci\%C3\%B3n-y-uso-del-plugin-Kobo-Touch-Extended-para-Calibre)
\item
  (http://www.redelijkheid.com/blog/2013/7/25/kobo-glo-ebook-library-management-with-calibre)
\item
  (https://www.netogram.com/kobo.htm)
\end{itemize}

\hypertarget{scribus}{%
\subsection{Scribus}\label{scribus}}

Scribus es un programa libre de composición de documentos. con Scribus
puedes elaborar desde los folletos de una exposición hasta una revista o
un poster.

Para tener una versión más actualizada instalamos:

\begin{verbatim}
sudo add-apt-repository ppa:scribus/ppa
sudo apt update
sudo apt install scribus scribus-ng scribus-template scribus-ng-doc
\end{verbatim}

\hypertarget{cambiados-algunos-valores-por-defecto}{%
\subsubsection{Cambiados algunos valores por
defecto}\label{cambiados-algunos-valores-por-defecto}}

He cambiado los siguientes valores en las dos versiones, non están
exactamente en el mismo menú pero no son díficiles de encontrar:

\begin{itemize}
\tightlist
\item
  Lenguaje por defecto: \textbf{English}
\item
  Tamaño de documento: \textbf{A4}
\item
  Unidades por defecto: \textbf{milimeters}
\item
  Show Page Grid: \textbf{Activado}
\item
  Dimensiones de la rejilla:

  \begin{itemize}
  \tightlist
  \item
    Mayor: \textbf{30 mm}
  \item
    Menor: \textbf{6mm}
  \end{itemize}
\item
  En opciones de salida de \emph{pdf} indicamos que queremos salida a
  impresora y no a pantalla. Y también que no queremos \emph{spot
  colors} (que serían solo para ciertas impresoras industriales).
\end{itemize}

Siempre se puede volver a los valores por defecto sin mucho problema
(hay una opción para ello)

Referencia
\href{https://www.youtube.com/watch?v=3sEoYZGABQM\&list=PL3kOqLpV3a67b13TY3WxYVzErYUOLYekI}{aquí}

\hypertarget{solucionados-problemas-de-hyphenation}{%
\subsubsection{\texorpdfstring{Solucionados problemas de
\emph{hyphenation}}{Solucionados problemas de hyphenation}}\label{solucionados-problemas-de-hyphenation}}

\emph{Scribus} no hacia correctamente la separación silábica en
castellano, he instalado los paquetes:

\begin{itemize}
\tightlist
\item
  hyphen-es
\item
  hyphen-gl
\end{itemize}

Y ahora funciona correctamente.

\hypertarget{desarrollo-software}{%
\section{Desarrollo software}\label{desarrollo-software}}

\hypertarget{paquetes-esenciales}{%
\subsection{Paquetes esenciales}\label{paquetes-esenciales}}

Estos son los paquetes esenciales para empezar a desarrollar software en
Linux.

\begin{verbatim}
sudo apt install build-essential checkinstall make automake cmake autoconf git git-core dpkg wget
\end{verbatim}

\hypertarget{git}{%
\subsection{Git}\label{git}}

Control de versiones distribuido. Imprescindible. Para \emph{Linux Mint}
viene instalado por defecto.

Configuración básica de git:

\begin{verbatim}
git config --global ui.color auto
git config --global user.name "Pepito Pérez"
git config --global user.email "pperez@mikasa.com"

git config --global alias.cl clone

git config --global alias.st "status -sb"
git config --global alias.last "log -1 --stat"
git config --global alias.lg "log --graph --pretty=format:'%Cred%h%Creset -%C(yellow)%d%Creset %s %Cgreen(%cr) %Cblue<%an>%Creset' --abbrev-commit --date=relative --all"
git config --global alias.dc "diff --cached"

git config --global alias.unstage "reset HEAD --"

git config --global alias.ci commit
git config --global alias.ca "commit -a"

git config --global alias.ri "rebase -i"
git config --global alias.ria "rebase -i --autosquash"
git config --global alias.fix "commit --fixup"
git config --global alias.squ "commit --squash"

git config --global alias.cp cherry-pick
git config --global alias.co checkout
git config --global alias.br branch
git config --global core.editor emacs
\end{verbatim}

\hypertarget{emacs}{%
\subsection{Emacs}\label{emacs}}

Instalado emacs desde los repos:

\begin{verbatim}
sudo aptitude install emacs
\end{verbatim}

\begin{itemize}
\tightlist
\item
  Configuramos la fuente por defecto del editor y salvamos las opciones.
  Con esto generamos el fichero \texttt{\textasciitilde{}/.emacs}
\item
  \textbf{Importante}: Configuramos la \emph{face} para la \emph{region}
  con un color que nos guste. Parece que viene configurado por defecto
  igual que el texto normal y nunca veremos la \emph{region} resaltada
  aunque queramos.
\item
  Editamos el fichero \texttt{.emacs} y añadimos los depósitos de
  paquetes (nunca he conseguido que \emph{Marmalade} funcione)
\end{itemize}

Esta es la sección donde configuramos los depósitos de paquetes:

\begin{verbatim}
;;----------------------------------------------------------------------
;; MELPA and others
(when (>= emacs-major-version 24)
  (require 'package)
  (package-initialize)
  (add-to-list 'package-archives '("melpa" . "http://melpa.org/packages/") t)
  (add-to-list 'package-archives '("gnu" . "http://elpa.gnu.org/packages/") t)
;;  (add-to-list 'package-archives '("marmalade" . "https://marmalade-repo.org/packages/") t)
  )
\end{verbatim}

GNU Elpa es el depósito oficial, tiene menos paquetes y son todos con
licencia FSF.

Melpa y Marmalade son paquetes de terceros. Tienen mucha más variedad
pero con calidades dispares.

Desde Melpa con el menú de gestión de paquetes de emacs, instalamos los
siguientes paquetes:

\begin{itemize}
\tightlist
\item
  \emph{markdown-mode}
\item
  \emph{pandoc-mode}
\item
  \emph{auto-complete}
\item
  \emph{ac-dcd}
\item
  \emph{d-mode}
\item
  \emph{flycheck}
\item
  \emph{flycheck-dmd-dub}
\item
  \emph{flycheck-d-unittest}
\item
  \emph{elpy}
\item
  \emph{jedi}
\end{itemize}

Después de probar \emph{flymake} y \emph{flycheck} al final me ha
gustado más \emph{flycheck} Hay una sección de configuración en el
fichero \texttt{.emacs} para cada uno de ellos, pero la de
\emph{flymake} está comentada.

Configuramos el fichero \texttt{.emacs} definimos algunas preferencias,
algunas funciones útiles y añadimos orígenes extra de paquetes.

\begin{verbatim}
(custom-set-variables
 ;; custom-set-variables was added by Custom.
 ;; If you edit it by hand, you could mess it up, so be careful.
 ;; Your init file should contain only one such instance.
 ;; If there is more than one, they won't work right.
 '(show-paren-mode t))
(custom-set-faces
 ;; custom-set-faces was added by Custom.
 ;; If you edit it by hand, you could mess it up, so be careful.
 ;; Your init file should contain only one such instance.
 ;; If there is more than one, they won't work right.
 )

;;------------------------------------------------------------
;; Some settings
(setq inhibit-startup-message t) ; Eliminate FSF startup msg
(setq frame-title-format "%b")   ; Put filename in titlebar
;(setq visible-bell t)           ; Flash instead of beep
(set-scroll-bar-mode 'right)     ; Scrollbar placement
(show-paren-mode t)              ; Blinking cursor shows matching parentheses
(setq column-number-mode t)      ; Show column number of current cursor location
(mouse-wheel-mode t)             ; wheel-mouse support

(setq fill-column 78)
(setq auto-fill-mode t)          ; Set line width to 78 columns...

(setq-default indent-tabs-mode nil)       ; Insert spaces instead of tabs
(global-set-key "\r" 'newline-and-indent) ; turn autoindenting on
;(set-default 'truncate-lines t)          ; Truncate lines for all buffers
;(require 'iso-transl)                    ; doesn't seems to be needed in debian


;;------------------------------------------------------------
;; Some useful key definitions
(define-key global-map [M-S-down-mouse-3] 'imenu)
(global-set-key [C-tab] 'hippie-expand)                    ; expand
(global-set-key [C-kp-subtract] 'undo)                     ; [Undo]
(global-set-key [C-kp-multiply] 'goto-line)                ; goto line
(global-set-key [C-kp-add] 'toggle-truncate-lines)         ; goto line
(global-set-key [C-kp-divide] 'delete-trailing-whitespace) ; delete trailing whitespace
(global-set-key [C-kp-decimal] 'completion-at-point)       ; complete at point
(global-set-key [C-M-prior] 'next-buffer)                  ; next-buffer
(global-set-key [C-M-next] 'previous-buffer)               ; previous-buffer

;;------------------------------------------------------------
;; Set encoding
(prefer-coding-system 'utf-8)
(setq coding-system-for-read 'utf-8)
(setq coding-system-for-write 'utf-8)

;;------------------------------------------------------------
;; Maximum colors
(cond ((fboundp 'global-font-lock-mode)        ; Turn on font-lock (syntax highlighting)
       (global-font-lock-mode t)               ; in all modes that support it
       (setq font-lock-maximum-decoration t))) ; Maximum colors

;;------------------------------------------------------------
;; Use % to match various kinds of brackets...
;; See: http://www.lifl.fr/~hodique/uploads/Perso/patches.el

(global-set-key "%" 'match-paren)               ; % key match parents
(defun match-paren (arg)
  "Go to the matching paren if on a paren; otherwise insert %."
  (interactive "p")
  (let ((prev-char (char-to-string (preceding-char)))
        (next-char (char-to-string (following-char))))
    (cond ((string-match "[[{(<]" next-char) (forward-sexp 1))
          ((string-match "[\]})>]" prev-char) (backward-sexp 1))
          (t (self-insert-command (or arg 1))))))

;;------------------------------------------------------------
;; The wonderful bubble-buffer
(defvar LIMIT 1)
(defvar time 0)
(defvar mylist nil)

(defun time-now ()
   (car (cdr (current-time))))

(defun bubble-buffer ()
   (interactive)
   (if (or (> (- (time-now) time) LIMIT) (null mylist))
       (progn (setq mylist (copy-alist (buffer-list)))
          (delq (get-buffer " *Minibuf-0*") mylist)
          (delq (get-buffer " *Minibuf-1*") mylist)))
   (bury-buffer (car mylist))
   (setq mylist (cdr mylist))
   (setq newtop (car mylist))
   (switch-to-buffer (car mylist))
   (setq rest (cdr (copy-alist mylist)))
   (while rest
     (bury-buffer (car rest))
     (setq rest (cdr rest)))
   (setq time (time-now)))

(global-set-key [f8] 'bubble-buffer)    ; win-tab switch the buffer

(defun geosoft-kill-buffer ()
   ;; Kill default buffer without the extra emacs questions
   (interactive)
   (kill-buffer (buffer-name))
   (set-name))
(global-set-key [C-delete] 'geosoft-kill-buffer)

;;----------------------------------------------------------------------
;; MELPA and others
(when (>= emacs-major-version 24)
  (require 'package)
  (package-initialize)
  (add-to-list 'package-archives '("melpa" . "http://melpa.org/packages/") t)
  (add-to-list 'package-archives '("gnu" . "http://elpa.gnu.org/packages/") t)
  (add-to-list 'package-archives '("marmalade" . "https://marmalade-repo.org/packages/") t)
  )

; (add-to-list 'load-path "~/.emacs.d/")

;;----------------------------------------------------------------------
;; Packages installed via package
;;------------------------------

;;----------------------------------------------------------------------
;; flymake and flycheck installed from package
;; I think you have to choose only one

;; (require 'flymake)
;; ;;(global-set-key (kbd "C-c d") 'flymake-display-err-menu-for-current-line)
;; (global-set-key (kbd "C-c d") 'flymake-popup-current-error-menu)
;; (global-set-key (kbd "C-c n") 'flymake-goto-next-error)
;; (global-set-key (kbd "C-c p") 'flymake-goto-prev-error)

(add-hook 'after-init-hook #'global-flycheck-mode)
(global-set-key  (kbd "C-c C-p") 'flycheck-previous-error)
(global-set-key  (kbd "C-c C-n") 'flycheck-next-error)

;; Define d-mode addons
;; Activate flymake or flycheck for D
;; Activate auto-complete-mode
;; Activate yasnippet minor mode if available
;; Activate dcd-server
(require 'ac-dcd)
(add-hook 'd-mode-hook
          (lambda()
            ;;(flymake-d-load)
            (flycheck-dmd-dub-set-variables)
            (require 'flycheck-d-unittest)
            (setup-flycheck-d-unittest)
            (auto-complete-mode t)
            (when (featurep 'yasnippet)
              (yas-minor-mode-on))
            (ac-dcd-maybe-start-server)
            (ac-dcd-add-imports)
            (add-to-list 'ac-sources 'ac-source-dcd)
            (define-key d-mode-map (kbd "C-c ?") 'ac-dcd-show-ddoc-with-buffer)
            (define-key d-mode-map (kbd "C-c .") 'ac-dcd-goto-definition)
            (define-key d-mode-map (kbd "C-c ,") 'ac-dcd-goto-def-pop-marker)
            (define-key d-mode-map (kbd "C-c s") 'ac-dcd-search-symbol)
            (when (featurep 'popwin)
              (add-to-list 'popwin:special-display-config
                           `(,ac-dcd-error-buffer-name :noselect t))
              (add-to-list 'popwin:special-display-config
                           `(,ac-dcd-document-buffer-name :position right :width 80))
              (add-to-list 'popwin:special-display-config
                           `(,ac-dcd-search-symbol-buffer-name :position bottom :width 5)))))

;; Define diet template mode (this is not installed from package)
(add-to-list 'auto-mode-alist '("\\.dt$" . whitespace-mode))
(add-hook 'whitespace-mode-hook
          (lambda()
            (setq tab-width 2)
            (setq whitespace-line-column 250)
            (setq indent-tabs-mode nil)
            (setq indent-line-function 'insert-tab)))

;;----------------------------------------------------------------------
;; elpy
(elpy-enable)
\end{verbatim}

\hypertarget{lenguaje-de-programaciuxf3n-d-d-programming-language}{%
\subsection{Lenguaje de programación D (D programming
language)}\label{lenguaje-de-programaciuxf3n-d-d-programming-language}}

El lenguaje de programación D es un lenguaje de programación de sistemas
con una sintaxis similar a la de C y con tipado estático. Combina
eficiencia, control y potencia de modelado con seguridad y
productividad.

\hypertarget{d-apt-e-instalaciuxf3n-de-programas}{%
\subsubsection{D-apt e instalación de
programas}\label{d-apt-e-instalaciuxf3n-de-programas}}

Configurado \emph{d-apt}, instalados todos los programas incluidos

\begin{verbatim}
sudo wget http://master.dl.sourceforge.net/project/d-apt/files/d-apt.list -O /etc/apt/sources.list.d/d-apt.list
sudo apt-key adv --keyserver keyserver.ubuntu.com --recv-keys  EBCF975E5BA24D5E
sudo apt update
\end{verbatim}

Instalamos todos los programas asociados excepto \emph{textadept} que
falla por problemas de librerias.

\begin{verbatim}
sudo apt install dmd-compiler dmd-tools dub dcd dfix dfmt dscanner
\end{verbatim}

\hypertarget{dcd}{%
\subsubsection{DCD}\label{dcd}}

Una vez instalado el DCD tenemos que configurarlo creando el fichero
\texttt{\textasciitilde{}/.config/dcd/dcd.conf} con el siguiente
contenido:

\begin{verbatim}
/usr/include/dmd/druntime/import
/usr/include/dmd/phobos
\end{verbatim}

Podemos probarlo con:

\begin{verbatim}
dcd-server &
echo | dcd-client --search toImpl
\end{verbatim}

\hypertarget{gdc}{%
\subsubsection{gdc}\label{gdc}}

Instalado con:

\begin{verbatim}
sudo aptitude install gdc
\end{verbatim}

\hypertarget{ldc}{%
\subsubsection{ldc}\label{ldc}}

Instalado con:

\begin{verbatim}
sudo aptitude install ldc
\end{verbatim}

Para poder ejecutar aplicaciones basadas en Vibed, necesitamos instalar:

\begin{verbatim}
sudo apt-get install -y libssl-dev libevent-dev
\end{verbatim}

\hypertarget{emacs-para-editar-d}{%
\subsubsection{Emacs para editar D}\label{emacs-para-editar-d}}

Instalados los siguientes paquetes desde Melpa

\begin{itemize}
\tightlist
\item
  d-mode
\item
  flymake-d
\item
  flycheck
\item
  flycheck-dmd-dub
\item
  flychek-d-unittest
\item
  auto-complete (desde melpa)
\item
  ac-dcd
\end{itemize}

Referencias * (https://github.com/atilaneves/ac-dcd) *
(https://github.com/Hackerpilot/DCD)

\hypertarget{processing}{%
\subsection{Processing}\label{processing}}

Bajamos los paquetes de las respectivas páginas web, descomprimimimos en
\texttt{\textasciitilde{}/apps/}, en las nuevas versiones incorpora un
script de instalación que ya se encarga de crear el fichero
\emph{desktop}.

\hypertarget{python}{%
\subsection{Python}\label{python}}

De partida tenemos instalado dos versiones: \emph{python} y
\emph{python3}

\begin{verbatim}
python -V
Python 2.7.12

python3 -V
Python 3.5.2
\end{verbatim}

\hypertarget{paquetes-de-desarrollo}{%
\subsubsection{Paquetes de desarrollo}\label{paquetes-de-desarrollo}}

Para que no haya problemas a la hora de instalar paquetes en el futuro
conviene que instalemos los paquetes de desarrollo:

\begin{verbatim}
sudo apt install python-dev
sudo apt install python3-dev
\end{verbatim}

\hypertarget{pip-virtualenv-virtualenvwrapper-virtualfish}{%
\subsubsection{pip, virtualenv, virtualenvwrapper,
virtualfish}\label{pip-virtualenv-virtualenvwrapper-virtualfish}}

Los he instalado a nivel de sistema.

\emph{pip} es un gestor de paquetes para \textbf{Python} que facilita la
instalación de librerías y utilidades.

Para poder usar los entornos virtuales instalaremos también
\emph{virtualenv}.

Instalamos los dos desde aptitude:

\begin{verbatim}
sudo apt install python-pip python-virtualenv virtualenv python3-pip
\end{verbatim}

\emph{virtualenv} es una herramienta imprescindible en Python, pero da
un poco de trabajo, así que se han desarrollado algunos frontends para
simplificar su uso, para \emph{bash} y \emph{zsh} usaremos
\emph{virtualenvwrapper}, y para \emph{fish} el \emph{virtualfish}. Como
veremos son todos muy parecidos.

Instalamos el virtualwrapper:

\begin{verbatim}
sudo apt install virtualenvwrapper -y
\end{verbatim}

Para usar \emph{virtualenvwrapper} tenemos que hacer:

\begin{verbatim}
source /usr/share/virtualenvwrapper/virtualenvwrapper.sh
\end{verbatim}

O añadir esa linea a nuestros ficheros \emph{.bashrc} y/o \emph{.zshrc}

Definimos la variable de entorno \emph{WORKON\_HOME} para que apunte al
directorio por defecto,
\texttt{\textasciitilde{}/.local/share/virtualenvs}. En ese directorio
es donde se guardarán nuestros entornos virtuales.

En el fichero \texttt{.profile} añadimos:

\begin{verbatim}
# WORKON_HOME for virtualenvwrapper
if [ -d "$HOME/.local/share/virtualenvs" ] ; then
    WORKON_HOME="$HOME/.local/share/virtualenvs"
fi
\end{verbatim}

\href{http://virtualenvwrapper.readthedocs.io/en/latest/command_ref.html}{Aquí}
tenemos la referencia de comandos de \emph{virtualenvwrapper}

Por último, si queremos tener utilidades parecidas en nuestro \emph{fish
shell} instalamos \emph{virtualfish}:

\begin{verbatim}
sudo pip install virtualfish
\end{verbatim}

\href{http://virtualfish.readthedocs.io/en/latest/index.html}{Aquí}
tenemos la documentación de \emph{virtualfish} y la descripción de todos
los comandos y plugins disponibles.

\hypertarget{pipenv}{%
\subsubsection{pipenv}\label{pipenv}}

No lo he instalado, pero en caso de instalación mejor lo instalamos a
nivel de usuario con:

\begin{verbatim}
pip install --user pipenv
\end{verbatim}

\hypertarget{instalaciuxf3n-de-bpython-y-ptpython}{%
\subsubsection{Instalación de bpython y
ptpython}\label{instalaciuxf3n-de-bpython-y-ptpython}}

\emph{bpython} instalado desde repos
\texttt{sudo\ apt\ install\ bpython\ bpython3}

\emph{ptpython} instalado en un virtualenv para probarlo

\hypertarget{emacs-para-programar-python}{%
\subsubsection{Emacs para programar
python}\label{emacs-para-programar-python}}

Para instalar \texttt{elpy}

\begin{verbatim}
sudo apt install python-jedi python3-jedi
# flake8 for code checks
sudo apt install flake8 python-flake8 python3-flake8
# and autopep8 for automatic PEP8 formatting
sudo apt install python-autopep8
# and yapf for code formatting
sudo apt install yapf yapf3
\end{verbatim}

Añadimos la sección

\begin{verbatim}
;;----------------------------------------------------------------------
;; elpy
(elpy-enable)
(setq elpy-rpc-backend "jedi")

(add-hook 'python-mode-hook 'jedi:setup)
(setq jedi:complete-on-dot t)
\end{verbatim}

Desde \emph{Emacs} ejecutamos: \texttt{alt-x\ jedi:install-server}

\hypertarget{todo}{%
\paragraph{TODO}\label{todo}}

Estudiar esto con calma \url{https://elpy.readthedocs.io/en/latest}

\hypertarget{jupyter}{%
\subsubsection{Jupyter}\label{jupyter}}

Una instalación para pruebas.

\begin{verbatim}
mkvirtualenv -p /usr/bin/python3 jupyter
python -m pip install jupyter 
\end{verbatim}

\hypertarget{neovim}{%
\subsection{neovim}\label{neovim}}

Vamos a probar \emph{neovim}:

\begin{verbatim}
sudo apt-add-repository ppa:neovim-ppa/stable
sudo apt-get update
sudo apt-get install neovim
\end{verbatim}

Para instalar los módulos de python \footnote{aun no lo hice}:

\begin{verbatim}
sudo pip install --upgrade neovim
sudo pip3 install --upgrade neovim
\end{verbatim}

Revisar
\href{https://neovim.io/doc/user/provider.html\#provider-python}{esto}

Para actualizar las alternativas:

\begin{verbatim}
sudo update-alternatives --install /usr/bin/vi vi /usr/bin/nvim 60
sudo update-alternatives --config vi
sudo update-alternatives --install /usr/bin/vim vim /usr/bin/nvim 60
sudo update-alternatives --config vim
\end{verbatim}

\hypertarget{install-vim-plug}{%
\paragraph{\texorpdfstring{Install
\emph{vim-plug}}{Install vim-plug}}\label{install-vim-plug}}

Ejecutamos:

\begin{verbatim}
curl -fLo ~/.local/share/nvim/site/autoload/plug.vim --create-dirs \
    https://raw.githubusercontent.com/junegunn/vim-plug/master/plug.vim
\end{verbatim}

Configuramos el fichero de configuración de \emph{nvim}
(\texttt{\textasciitilde{}/.config/nvim/init.vim}):

\begin{verbatim}
" Specify a directory for plugins
" - For Neovim: ~/.local/share/nvim/plugged
" - Avoid using standard Vim directory names like 'plugin'
call plug#begin('~/.local/share/nvim/plugged')

if has('nvim')
  Plug 'Shougo/deoplete.nvim', { 'do': ':UpdateRemotePlugins' }
else
  Plug 'Shougo/deoplete.nvim'
  Plug 'roxma/nvim-yarp'
  Plug 'roxma/vim-hug-neovim-rpc'
endif
let g:deoplete#enable_at_startup = 1


Plug 'zchee/deoplete-jedi'

" Initialize plugin system
call plug#end()
\end{verbatim}

La primera vez que abramos \emph{nvim} tenemos que instalar los plugin
por comando ejecutando: \texttt{:PlugInstall}

\textbf{Instalación de \texttt{dein}}

Solo hay que instalar uno de los dos o \emph{dein} o \emph{plug-vim}. Yo
uso \emph{plug-vim} así que esto es sólo una referencia.

\url{https://github.com/Shougo/dein.vim}

\begin{verbatim}
" Add the dein installation directory into runtimepath
set runtimepath+=~/.config/nvim/dein/repos/github.com/Shougo/dein.vim

if dein#load_state('~/.config/nvim/dein')
  call dein#begin('~/.config/nvim/dein')

  call dein#add('~/.config/nvim/dein/repos/github.com/Shougo/dein.vim')
  call dein#add('Shougo/deoplete.nvim')
  call dein#add('Shougo/denite.nvim')
  if !has('nvim')
    call dein#add('roxma/nvim-yarp')
    call dein#add('roxma/vim-hug-neovim-rpc')
  endif

  call dein#end()
  call dein#save_state()
endif

filetype plugin indent on
syntax enable
\end{verbatim}

\hypertarget{desarrollo-hardware}{%
\section{Desarrollo hardware}\label{desarrollo-hardware}}

\hypertarget{arduino-ide}{%
\subsection{Arduino IDE}\label{arduino-ide}}

Bajamos los paquetes de la página \href{https://www.arduino.cc}{web},
descomprimimimos en \emph{\textasciitilde/apps/arduino}.

La distribución del IDE incluye ahora un fichero que se encarga de hacer
la integración del IDE en los menús de Linux.

Hay que añadir nuestro usuario al grupo \texttt{dialout}:

\begin{verbatim}
sudo gpasswd --add <usrname> dialout
\end{verbatim}

\hypertarget{auxf1adir-biblioteca-de-soporte-para-makeblock}{%
\subsubsection{Añadir biblioteca de soporte para
Makeblock}\label{auxf1adir-biblioteca-de-soporte-para-makeblock}}

Clonamos el
\href{https://github.com/Makeblock-official/Makeblock-Libraries}{repo
oficial en github}.

Una vez que descarguemos las librerias es necesario copiar el directorio
\texttt{Makeblock-Libraries/makeblock} en nuestro directorio de
bibliotecas de Arduino. En mi caso
\texttt{\textasciitilde{}/Arduino/libraries/}.

Una vez instaladas las bibliotecas es necesario reiniciar el IDE Arduino
si estaba arrancado. Podemos ver si se ha instalado correctamente
simplemente echando un ojo al menú de ejemplos en el IDE, tendríamos que
ver los ejemplos de \emph{Makeblock}.

Un detalle importante para programar el Auriga-Me es necesario
seleccionar el micro Arduino Mega 2560 en el IDE Arduino.

\hypertarget{pinguino-ide}{%
\subsection{Pinguino IDE}\label{pinguino-ide}}

\begin{longtable}[]{@{}l@{}}
\toprule
\endhead
\textbf{Nota}: Pendiente de instalar\tabularnewline
\bottomrule
\end{longtable}

Tenemos el paquete de instalación disponible en su página
\href{http://pinguino.cc/download.php}{web}

Ejecutamos el programa de instalación. El programa descargará los
paquetes Debian necesarios para dejar el IDE y los compiladores
instalados.

Al acabar la instalación he tenido que crear el directorio
\emph{\textasciitilde/Pinguino/v11}, parece que hay algún problema con
el programa de instalación y no lo crea automáticamente.

El programa queda correctamente instalado en \emph{/opt} y arranca
correctamente, habrá que probarlo con los micros.

\hypertarget{kicad}{%
\subsection{KiCAD}\label{kicad}}

En la \href{http://kicad-pcb.org/download/linux-mint/}{página web del
proyecto} nos recomiendan el ppa a usar para instalar la última versión
estable:

\begin{verbatim}
sudo add-apt-repository --yes ppa:js-reynaud/kicad-5
sudo apt-get update
sudo apt-get install kicad
sudo apt install kicad-footprints kicad-libraries kicad-packages3d kicad-symbols kicad-templates
\end{verbatim}

Paciencia, el paquete \texttt{kicad-packages3d} tarda un buen rato en
descargarse.

Algunas librerías alternativas:

\begin{itemize}
\tightlist
\item
  \href{https://github.com/freetronics/freetronics_kicad_library}{Freetronics}
  una libreria que no solo incluye Shield para Arduino sino una completa
  colección de componentes que nos permitirá hacer proyectos completos.
  \href{http://www.freetronics.com}{Freetronics} es una especie de
  BricoGeek australiano, publica tutoriales, vende componentes, y al
  parecer mantiene una biblioteca para KiCAD. La biblioteca de
  Freetronics se mantiene en un repo de github. Lo suyo es incorporarla
  a cada proyecto, por que si la actualizas se pueden romper los
  proyectos que estes haciendo.
\item
  \href{http://meta-blog.eklablog.com/kicad-librairie-arduino-pretty-p930786}{eklablog}
  Esta biblioteca de componentes está incluida en el github de KiCAD,
  así que teoricamente no habría que instalarla en nuestro disco duro.
\end{itemize}

\hypertarget{analizador-luxf3gico}{%
\subsection{Analizador lógico}\label{analizador-luxf3gico}}

Mi analizador es un OpenBench de Seedstudio,
\href{http://dangerousprototypes.com/docs/Open_Bench_Logic_Sniffer}{aquí
hay mas info}

\hypertarget{sigrok}{%
\subsubsection{Sigrok}\label{sigrok}}

Instalamos \textbf{Sigrok}, simplemente desde los repos de Debian:

\begin{verbatim}
sudo aptitude install sigrok
\end{verbatim}

Al instalar \textbf{Sigrok} instalamos también \textbf{Pulseview}.

Si al conectar el analizador, echamos un ojo al fichero \emph{syslog}
vemos que al conectarlo se mapea en un puerto tty.

Si arrancamos \textbf{Pulseview} (nuestro usuario tiene que estar
incluido en el grupo \emph{dialout}), en la opción \emph{File::Connect
to device}, escogemos la opción \emph{Openbench} y le pasamos el puerto.
Al pulsar la opción \emph{Scan for devices} reconoce el analizador
correctamente como un \emph{Sump Logic Analyzer}.

\hypertarget{sump-logic-analyzer}{%
\subsubsection{Sump logic analyzer}\label{sump-logic-analyzer}}

Este es el software recomendado para usar con el analizador.

Descargamos el paquete de la \href{https://www.sump.org}{página del
proyecto}, o más concretamente de
\href{https://www.sump.org/projects/analyzer/}{esta página} y
descomprimimos en \emph{\textasciitilde/apps}.

Instalamos las dependencias:

\begin{verbatim}
sudo apt install librxtx-java
\end{verbatim}

Editamos el fichero \emph{\textasciitilde/apps/Logic
Analyzer/client/run.sh} y lo dejamos así:

\begin{verbatim}
#!/bin/bash

# java -jar analyzer.jar $*
java -cp /usr/share/java/RXTXcomm.jar:analyzer.jar org.sump.analyzer.Loader
\end{verbatim}

Y ya funciona.

\hypertarget{ols}{%
\subsubsection{OLS}\label{ols}}

\begin{longtable}[]{@{}l@{}}
\toprule
\endhead
\textbf{Nota}: Pendiente de instalar\tabularnewline
\bottomrule
\end{longtable}

\href{https://www.lxtreme.nl/ols/}{Página oficial}

\hypertarget{icestudio}{%
\subsection{IceStudio}\label{icestudio}}

Instalamos dependencias con \texttt{sudo\ apt\ install\ xclip}

Bajamos el \emph{AppImage} desde el
\href{https://github.com/FPGAwars/icestudio}{github de IceStudio} y lo
dejamos en \texttt{\textasciitilde{}/apps/icestudio}

\hypertarget{platformio}{%
\subsection{PlatformIO}\label{platformio}}

Nos bajamos el paquete para instalar el \href{https://atom.io/}{Atom
IDE}

Instalamos el paquete \texttt{.deb} que nos hemos bajado:

\begin{verbatim}
sudo apt deb atom-amd64.deb
\end{verbatim}

Instalamos \texttt{clang} (una dependencia de PlatformIO)

\begin{verbatim}
sudo apt install clang
\end{verbatim}

Completamos la instalación del paquete desde el Atom siguiendo las
instrucciones
\href{https://platformio.org/get-started/ide?install=atom}{aquí}.

Para poder usar \emph{platformio} sin usar \emph{Atom} añadimos la
siguiente linea al fichero \texttt{.profile}:

\begin{verbatim}
export PATH=$PATH:~/.platformio/penv/bin
\end{verbatim}

\begin{itemize}
\tightlist
\item
  \href{https://docs.platformio.org/en/latest/installation.html\#piocore-install-shell-commands}{Referencia}
\end{itemize}

\hypertarget{reprap}{%
\subsection{RepRap}\label{reprap}}

\hypertarget{openscad}{%
\subsubsection{OpenScad}\label{openscad}}

El OpenSCAD disponible en los orígenes de software parece ser la última
estable. Así que instalamos con \texttt{apt}:

\begin{verbatim}
sudo apt install openscad
\end{verbatim}

\hypertarget{slic3r}{%
\subsubsection{Slic3r}\label{slic3r}}

Descargamos la estable desde la \href{https://dl.slic3r.org}{página web}
y como de costumbre descomprimimos en \texttt{\textasciitilde{}/apps} y
creamos un lanzador con \emph{MenuLibre}

\hypertarget{slic3r-prusa-edition}{%
\subsubsection{Slic3r Prusa Edition}\label{slic3r-prusa-edition}}

Una nueva versión del clásico \emph{Slic3r} con muchas mejoras.
Descargamos la \emph{appimage} desde la
\href{https://www.prusa3d.com/slic3r-prusa-edition/}{página web} y ya
sabeis, descomprimir en \texttt{\textasciitilde{}/apps} y dar permisos
de ejecución.

\hypertarget{ideamaker}{%
\subsubsection{ideaMaker}\label{ideamaker}}

Una aplicación más para generar gcode con muy buena pinta, tenemos el
paquete \emph{deb} disponible en su
\href{https://www.raise3d.com/pages/ideamaker}{página web}. Instalamos
con el gestor de software.

\hypertarget{ultimaker-cura}{%
\subsubsection{Ultimaker Cura}\label{ultimaker-cura}}

Descargamos el \emph{AppImage} desde la
\href{https://github.com/mypaint/mypaint/releases}{página web}

\hypertarget{pronterface}{%
\subsubsection{Pronterface}\label{pronterface}}

Seguimos las instrucciones para Ubuntu Bionic:

Instalamos las dependencias:

\begin{verbatim}
sudo apt install python3-serial python3-numpy cython3 python3-libxml2 \
python3-gi python3-dbus python3-psutil python3-cairosvg libpython3-dev \
python3-appdirs python3-wxgtk4.0
\end{verbatim}

Seguimos
\href{https://github.com/kliment/Printrun/tree/master\#running-from-source}{las
instrucciones} para instalar desde los fuentes:

Clonamos el github:

\begin{verbatim}
cd ~/apps
git clone https://github.com/kliment/Printrun.git
\end{verbatim}

Nos hacemos un \emph{virtualenv}:

\begin{verbatim}
mkvirtualenv -p /usr/bin/python3 printrun

pip install  -f https://extras.wxpython.org/wxPython4/extras/linux/gtk3/ubuntu-16.04  wxPython
\end{verbatim}

Instalamos el resto de las dependencias con
\texttt{pip\ install\ -r\ requirements.txt}, o si lo hacemos a mano
sería:

\begin{verbatim}
pip install Cython
pip install pyserial
pip install numpy pyglet
pip install cffi
pip install cairocffi
pip install cairosvg
pip install psutil
pip install lxml
pip install appdirs
pip install pyreadline
pip install pyobjc-framework-Cocoa
pip install dbus-python
\end{verbatim}

A mi no me instala el dbus. Teóricamente es para impedir que el pc se
suspenda.

Ademas tenemos que ejecutar:

\begin{verbatim}
python setup.py build_ext --inplace
\end{verbatim}

\end{document}
